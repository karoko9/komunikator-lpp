\documentclass[a4paper,12pt]{article}
\usepackage{fancyhdr}
\usepackage[MeX]{polski}
\usepackage[utf8]{inputenc}
\usepackage{amsmath,amsthm,amssymb}
\usepackage{graphicx}

\usepackage[lmargin=2.7cm]{geometry}

\pagestyle{fancy}
\lhead{Komunikator internetowy -- Dokumentacja}
\rhead{\bfseries}



\author{Adrian Chudziński, Przemysław Gospodarczyk}
\title{Komunikator internetowy -- Dokumentacja}

\begin{document}

\makeatletter
    \renewcommand\@seccntformat[1]{\csname the#1\endcsname.\quad}
    \renewcommand\numberline[1]{#1.\hskip0.7em}
\makeatother

\begin{titlepage}
\begin{center}

    \textsc{Licencjacki projekt programistyczny}\\[0.1cm]
    \textsc{IIUWr 2009/2010}\\[6cm]
    Adrian Chudziński, Przemysław Gospodarczyk\\[1cm]
    \textsc{\Large Komunikator internetowy}\\[0.25cm]
    \textsc{\large Dokumentacja}\\[8.675cm]

    {\footnotesize
    \begin{tabular}{| c | p{4cm} | p{4.25cm} | c | }
        \hline
        Wersja  &
        Zmiany  &
        Autorzy &
        Data    \\
        \hline
        1.0                                                                   &
        Pierwsza wersja                                                       &
        \par Adrian Chudziński \par Przemysław Gospodarczyk                                                  &
        2010-03-25                                                            \\
        \hline
    \end{tabular}
    }

\end{center}
\end{titlepage}

\break

\setcounter{page}{2}

\tableofcontents

\break
\section[Słownik pojęć]{Słownik pojęć}
\noindent\textbf{America Online} (AOL) -- największy amerykański dostawca usług internetowych. Jeden z pierwszych wydawców elektronicznego biuletynu informacyjnego (BBS) w USA, który od 1995 roku jest dostępny również w Internecie. W 1998 roku wykupił firmę Netscape -- producenta przeglądarki Netscape Navigator.\\

\noindent\textbf{API} (\textit{Application Programming Interface}) -- zbiór konwencji wywoływania funkcji określających sposób dostępu do danej usługi przez system operacyjny oraz lista dostępnych funkcji wraz z opisem znaczenia ich parametrów.\\

\noindent\textbf{Commodore 64} -- model popularnego w latach osiemdziesiątych domowego komputera 8-bitowego firmy Commodore. Oparty na procesorach 6510 lub 8502 (w zależności od wersji), wyposażony w 64 KB pamięci operacyjnej RAM oraz odrębne procesory przetwarzające dane wejściowe i wyjściowe, obraz oraz dźwięk, co odciążało jednostkę centralną i umożliwiało tworzenie wydajnego oprogramowania.\\

\noindent\textbf{Dzielona tablica} (ang. \textit{shared whiteboard}) -- miejsce na dysku dostępne dla wielu użytkowników jednocześnie. Wersje poszczególnych użytkowników są synchronizowane w czasie zbliżonym do rzeczywistego.\\

\noindent\textbf{Excite} -- wyszukiwarka internetowa autorstwa firmy o tej samej nazwie, która została w roku 1999 wykupiona przez amerykańską firmę At Home.\\

\noindent\textbf{Extensible Messaging and Presence Protocol} (dawniej Jabber) -- protokół komunikacji oraz powiadamiania o obecności w czasie rzeczywistym oparty na XML. Głównym zastosowaniem Extensible Messaging and Presence Protocol jest wymiana wiadomości w komunikatorach internetowych. Serwery XMPP umożliwiają także za pomocą tzw. transportów komunikację z użytkownikami innych protokołów, np. Gadu-Gadu, Tlen.pl, ICQ i MSN Messenger.
Protokół jest również wykorzystywany w systemie blogowania Jogger przez XMPP.\\

\noindent\textbf{Google Talk} -- komunikator internetowy i usługa VoIP amerykańskiej firmy Google. Google Talk miał premierę 24 sierpnia 2005 roku. Wygląd programu (kolorystyka, układ tabel, czcionki) jest stylizowany na interfejs graficzny poczty elektronicznej Gmail.\\

\noindent\textbf{Interfejs graficzny} (ang. \textit{graphical interface, GUI}) -- dominująca obecnie odmiana interfejsu użytkownika, w którym kontakt z komputerem i sterowanie programami odbywa się za pomocą okien, ikon, przycisków, suwaków i rozmaitych menu. Interfejs graficzny jest obsługiwany za pomocą myszy i sprzężonego z nią kursora, możliwe jest też używanie w interfejsie graficznym operacji klawiszowych (skrót klawiaturowy).\\

\noindent\textbf{Jabber} -- patrz Extensible Messaging and Presence Protocol.\\

\noindent\textbf{Komunikator internetowy} (ang. \textit{Instant messenger}) -- program komputerowy pozwalający na przesyłanie natychmiastowych komunikatów (ang. \textit{Instant messaging}) pomiędzy dwoma lub więcej komputerami, poprzez sieć Internet. Od poczty elektronicznej różni się tym, że oprócz samej wiadomości, przesyłane są także informacje o obecności użytkowników, co zwiększa znacznie szansę na prowadzenie bezpośredniej konwersacji.\\

\noindent\textbf{MSN} (ang. \textit{MicroSoft Network}) -- serwis internetowy firmy Microsoft, który oferuje pocztę elektroniczną oraz fora dyskusyjne dotyczące różnych dziedzin i tematów.\\

\noindent\textbf{PETSCII} (ang. \textit{PET Standard Code of Information Interchange}) -- rodzaj zbioru znaków ASCII z 1977 roku znany również jako CBM ASCII, używany w 8-bitowych komputerach Commodore 64.\\

\noindent\textbf{Pidgin} (dawniej GAIM) -- wieloplatformowy komunikator internetowy, obsługujący wiele protokołów transmisyjnych. Pidgin jest wolnym oprogramowaniem, dostępnym na warunkach GNU GPL i został stworzony przez Marka Spencera dla systemów uniksowych, jednak obecnie jest dostępny także dla systemów Windows, MacOS, SkyOS oraz Qt Extended (dla urządzeń PDA).\\

\noindent\textbf{Serwis społecznościowy (portal społecznościowy)} -- rodzaj interaktywnych stron WWW, które są współtworzone przez sieci społeczne osób podzielających wspólne zainteresowania lub chcących poznać zainteresowania innych. Większość portali społecznościowych dostarcza użytkownikom wielu sposobów komunikacji, np.: czaty, komunikatory, listy dyskusyjne, blogi i fora dyskusyjne.\\

\noindent\textbf{Sieć P2P} (ang. \textit{peer--to--peer}) -- równorzędna sieć typu każdy z każdym, architektura sieciowa oparta na równoważności wszystkich jej węzłów. W sieci P2P każdy komputer dysponuje podobnymi możliwościami oraz może inicjować połączenia. Nie ma ustalonej hierarchii ani centralnego serwera. Ten sam komputer może równocześnie pełnić rolę serwera i klienta, czyli pobierać dane z innych komputerów i udostępniać swoje zasoby wszystkim pozostałym komputerom.\\

\noindent\textbf{Twitter} -- darmowy serwis społecznościowy udostępniający usługę mikroblogowania umożliwiającą użytkownikom wysyłanie oraz odczytywanie tak zwanych tweetów. Tweet to krótka, nie przekraczająca 140 znaków wiadomość tekstowa wyświetlana na stronie użytkownika oraz dostarczana pozostałym użytkownikom, którzy obserwują dany profil. Użytkownicy mogą dodawać krótkie wiadomości do swojego profilu z poziomu strony głównej serwisu, wysyłając wiadomości SMS lub korzystając z zewnętrznych aplikacji.\\

\noindent\textbf{Ubique} -- firma produkująca oprogramowanie do błyskawicznej komunikacji. Najbardziej znaną aplikacją Ubique jest Virtual Places Chat, a technologia znalazła zastosowanie w aplikacji Lotus Sametime firmy IBM. Obecnie Ubique jest częścią IBM Haifa Labs.\\

\noindent\textbf{Yahoo} -- jeden z najpopularniejszych portali internetowych, należący do amerykańskiej firmy Yahoo. Oprócz hierarchicznego katalogu kategorii tematycznych zawiera także mechanizm przeszukiwania zasobów Internetu.\\

\noindent\textbf{Yahoo Messenger} -- komunikator internetowy firmy Yahoo. Pierwotnie został wydany 9 marca 1998 roku pod nazwą Yahoo Pager.\\

\section[Wstęp, komunikatory jako zjawisko społeczne]{Wstęp, komunikatory jako zjawisko społeczne}
Podstawowymi cechami Internetu są globalność i masowość.
Głównym elementem Internetu jest społeczność, która traktuje go nierzadko jako jedyne okno na świat.
Użytkownicy, korzystając z Internetu w poszukiwaniu informacji, danych lub szukając kontaktu z innymi ludźmi w świecie wirtualnym nie tylko rozwijają globalną sieć, ale stają się również jej integralną i nieodzowną częścią.
Człowiek jako istota społeczna uzewnętrznia swoją potrzebę komunikacji poprzez pośrednią i bezpośrednią wymianę informacji z innymi ludźmi. Komunikacja sieciowa jest pośrednia i pozbawiona komunikatów niewerbalnych takich jak:
ton głosu, mimika, wygląd i zachowanie. Wyżej wymienione aspekty komunikacji sieciowej jednoznacznie wskazują na jej prymitywność w stosunku do bezpośrednich kontaktów międzyludzkich. Niewerbalna komunikacja może wzmacniać, osłabiać, a nawet zaprzeczać przekazom werbalnym. W związku z tym komunikaty werbalne mają ogromne znaczenie w komunikacji i mogą stać się ważniejsze niż treść wypowiedzi. Internet powoduje, że wzajemne kontakty stają się bardziej powierzchowne i tracą ważną część swojego charakteru.

\par Komunikacja sieciowa posiada jednak swoje zalety, które przez wiele osób są marginalizowane.
Najważniejszą cechą komunikacji internetowej jest jej anonimowość. Użytkownicy często fałszują swoje dane osobowe lub nie podają ich w ogóle. Możliwa jest błyskawiczna zmiana tożsamości. Bezpośrednia komunikacja w świecie rzeczywistym jest znacznie trudniejsza niż internetowa. Ludzie hamowani przez nieśmiałość oraz lęk przed odrzuceniem w dużej mierze rezygnują z tego typu komunikacji. Anonimowość, często złudna powoduje, że wyżej wymienione czynniki tracą na znaczeniu. Badania japońskich naukowców z Ochanomizu University przeprowadzone na studentach dowodzą, że relacje interpersonalne w życiu wymagają pewnych umiejętności społecznych oraz orientacji na innych.

\par Komunikatory internetowe stały się bardzo popularną formą komunikacji wśród młodzieży, grupy społecznej, która najbardziej potrzebuje kontaktów z rówieśnikami oraz akceptacji i dowartościowania. Współczesna młodzież jest bardziej spragniona relacji interpersonalnych niż pokolenie urodzone w latach sześćdziesiątych i siedemdziesiątych.
Według Stevena Gerali, szefa chicagowskiego \emph{Department of Youth Ministry \& Adolescent Studies} współcześni młodzi ludzie chcą intymności, stałości oraz bezpiecznych związków z innymi. Przede wszystkim pragną być doceniani i chronieni. Ważnym czynnikiem staje się anonimowość, która zapewnia młodemu użytkownikowi bezpieczeństwo oraz do pewnego stopnia bezkarność i łatwość w poznawaniu oraz komunikacji z ludźmi z całego świata.
W Internecie zawiera się przyjaźnie, przeżywa miłość, zdobywa informacje i nowe hobby.\\
Nie każdy nastolatek jest wyposażony w wymagany przez komunikację bezpośrednią zestaw cech. Komunikatory internetowe ich nie wymagają. Każdy użytkownik może przedstawić siebie w zupełnie inny, wymarzony przez siebie sposób, koloryzując rzeczywistość. Nastolatkowie utrzymujący takie kontakty zwiększają swoją pewność siebie, która będzie procentować w życiu dorosłym.

\par Wbrew powszechnym opiniom, z komunikatorów internetowych i portali społecznościowych korzystają coraz częściej ludzie dorośli, których przeciąga darmowość usługi oraz prostota obsługi. W artykule \emph{Rosja też ćwierka}, z numeru 11. czasopisma \emph{Polityka}, autorzy Marek Ostrowski i Adam Szostkiewicz opisują portal społecznościowy \textbf{Twitter} jako miejsce wymiany zdań głównie między ludźmi pełnoletnimi.\\
Za niespodziankę można uznać wielką popularność Twittera, zważywszy na to, że umożliwia on wysyłanie wiadomości o długości do 140 znaków, co powinno stanowić poważne utrudnienie w korespondowaniu. Użytkownicy Twittera nie wykorzystują go jednak do poważnych rozmów, lecz do bezzwłocznego pisania o tym co ich w danym momencie interesuje i emocjonuje. Ponadto serwis pozwala na sprawdzenie, co interesuje innych oraz według psychologa z Uniwersytetu Warszawskiego Jana Zająca sprzyja rozmowom o niczym w celu podtrzymania relacji towarzyskich z innymi ludźmi.\\
Opinie o tym, że komunikacja internetowa cofa nas w czasie do ery sprzed rewolucji Gutenberga i powoduje kryzys prasy drukowanej są mocno przesadzone. Według pobieżnej analizy przeprowadzonej przez autorów wspomnianego artykułu, prasa jest ciągle bardziej opiniotwórcza i dostarcza więcej informacji.\\
Twitter sprzyja również rozwojowi literatury mądrości (ang. \textit{wisdom literature}), czyli pisaniu aforyzmów, czego nie można już uznać za prymitywną formę komunikacji. Wysyłane krótkich wiadomości nie tylko, więc zaśmieca głowę, ale również potrafi zmobilizować do myślenia.

\par Komunikatory internetowe pomagają w życiu społecznym osobom niepełnosprawnym, ludziom, którzy z powodu swojego kalectwa posiadają niskie zdanie o sobie oraz obawiają się odrzucenia z powodu swojej odmienności. W Internecie, chory człowiek nie obawia się zdemaskowania prawdy o sobie ani o swojej ewentualnej nieatrakcyjności fizycznej spowodowanej kalectwem.

\par Istnieje wiele teorii na temat wypierania kontaktów w świecie rzeczywistym przez kontakty w Internecie, a wpływ komunikatorów internetowych na życie społeczne człowieka jest oceniany przez naukowców bardzo niejednoznacznie.\\
Socjolog Sherry Turkle w swojej książce \emph{Life on the Screen} udowodniła, że rozwój komunikatorów internetowych i portali społecznościowych prowadzi najpierw do wyobcowania poszczególnych jednostek, a w konsekwencji do społecznego autyzmu i nadmiernego indywidualizmu internautów.
Według autorki internauci stają się leniwi, nie podejmują trudów związanych z utrzymywaniem kontaktów społecznych w świecie rzeczywistym.\\
Z drugiej strony, według kanadyjskiego socjologa Barry'ego Wellmana uznawanego powszechnie za jednego z najwybitniejszych badaczy współczesnej cywilizacji, w tym także Internetu, komunikacja sieciowa uzupełnia kontakty nie eliminując jednocześnie podstawowych form komunikacji, np. spotkań bezpośrednich. Według jego badań internaci częściej interesują się polityką i biorą udział w życiu społecznym.

\par Fakt, że w dobie elektronicznej pandemii gadulstwa autorzy projektu tworzą kolejne narzędzie do internetowego plotkowania należy usprawiedliwić ogromnym zainteresowaniem tego typu aplikacjami oraz, co za tym idzie ogromnymi pieniędzmi za reklamy. Przed napisaniem własnego oprogramowania, autorzy przeprowadzili analizę komunikatorów internetowych dostępnych na polskim rynku, aby nie powielać błędów konkurencji i wykorzystać najlepsze pomysły.

\section[Historia komunikatorów]{Historia komunikatorów}
Pomysł na komunikator błyskawicznie przesyłający i odbierający wiadomości jest starszy niż sama idea Internetu. Jest związany z wielodostępnymi systemami operacyjnymi z połowy lat sześćdziesiątych. Systemy operacyjne CTSS i Multics były używane do szybkiej i prostej komunikacji pomiędzy użytkownikami pracującymi na tej samej maszynie jednocześnie.

\par Wraz z pojawieniem się, a następnie rozwojem Internetu protokół wymiany wiadomości został poszerzony o możliwość wykorzystania globalnej sieci do przesyłania informacji. Pierwsze komunikatory sieciowe wydane w 1983 roku wykorzystywały protokół typu każdy z każdym (\textbf{sieć P2P}). Przykładami takich programów były: talk, ytalk oraz ntalk.
Niektóre komunikatory wykorzystywały architekturę klient-serwer. Przykładami aplikacji typu klient-serwer były: talker (1984) oraz IRC (1988).

\par Na fali popularności Bulletin Board System w latach osiemdziesiątych pojawiło się wiele serwisów komputerowych, udostępniających na maszynie jego właściciela miejsca, gdzie można umieszczać i czytać ogłoszenia, obsługiwać własną skrzynkę pocztową oraz pobierać i przesyłać pliki. Przykładem takiego serwisu jest Freelancin' Roundtable działający przez ponad 3 lata od października 1984 roku do listopada 1987 roku.

\par W drugiej połowie lat osiemdziesiątych i na początku lat dziewięćdziesiątych internetowy serwis Quantum Link oferował użytkownikom \textbf{Commodore 64} przesyłanie wiadomości między połączonymi użytkownikami. Wiadomości nazywane On-Line Messages (w skrócie OLM) pojawiały się na żółtym pasku wraz z informacją o nazwie nadawcy. Quantum Link wykorzystywał znaną z Commodore 64 grafikę tekstową \textbf{PETSCII}. \textbf{Interfejs graficzny} (o ile w ogóle może być o nim mowa w tym przypadku) nie był porównywalny z możliwościami późniejszych tego typu programów, które działały pod systemami Windows i Unix.\\
Późniejsza wersja Quantum Link znana jako \textbf{America Online} posiadała serwis do przesyłania wiadomości o nazwie AOL Instant Messenger (w skrócie AIM) i zyskała większą popularność niż poprzedniczka.

\par Pierwsze nowoczesne komunikatory internetowe wykorzystujące interfejsy graficzne, które wyglądem i możliwościami przypominały te znane z dzisiejszych czasów zaczęły powstawać od połowy lat dziewięćdziesiątych.\\
Przykładem takiej aplikacji jest PowWow (1998), który był jednym z pierwszych komunikatorów internetowych działających pod systemem Windows. Program jako pierwszy oferował wiele funkcjonalności, które w dzisiejszych czasach uznawane są za standardowe dla tego typu aplikacji. Ponadto zawierał wiele nietypowych i innowacyjnych dodatków takich jak:
\begin{itemize}
    \item[--] odtwarzacz plików dźwiękowych w formacie WAV;
    \item[--] wbudowany syntezator mowy;
    \item[--] program obsługujący protokół VoIP, który pozwalał na przesyłanie dźwięków mowy;
    \item[--] program obsługujący komunikację przez POP/SMTP;
    \item[--] program obsługujący \textbf{dzieloną tablicę} (ang. shared whiteboard), miejsce na dysku dostępne dla wielu użytkowników jednocześnie.
\end{itemize}
Innym przykładem jest działający również pod Windows ICQ (1996), wyprodukowany przez izraelską firmę Mirabilis. ICQ działa do dzisiaj, a w 1998 roku prawa do serwisu wykupił AOL za 407 mln. dolarów. W serwisie jest obecnie ponad 100 mln. zarejestrowanych kont, a AOL przyznano prawa do dwóch patentów dotyczących przesyłania informacji przez Internet. Angielski zwrot Instant Messenger, a także skrót IM, w Stanach Zjednoczonych jest własnością AOL i nie może być używany przez inne firmy. Działający pod wszystkimi najpopularniejszymi systemami operacyjnymi komunikator GAIM został z tego powodu przemianowany w 2007 roku na \textbf{Pidgin}.\\
W międzyczasie inne firmy pracowały nad swoim własnym oprogramowaniem, np. \textbf{Excite}, \textbf{MSN}, \textbf{Ubique} i \textbf{Yahoo}.
Wszystkie wymienione aplikacje posiadały własne, niezależne protokoły przesyłania wiadomości, co powodowało, że użytkownicy musieli korzystać ze wszystkich dostępnych aplikacji, aby móc połączyć się z każdą siecią.

\par W 1998 roku IBM wprowadził na rynek IBM Lotus Sametime oparty na technologii wykorzystywanej przez Ubique. Oprócz podstawowych, typowych funkcjonalności oprogramowanie IBM oferowało następujące nowości:
\begin{itemize}
    \item[--] przechowywanie historii rozmów;
    \item[--] możliwość przeprowadzania internetowych konferencji wideo;
    \item[--] chat;
    \item[--] możliwość wykorzystania kodeków graficznych i dźwiękowych;
    \item[--] możliwość połączenia się z innymi sieciami takimi jak: AOL Instant Messenger, \textbf{Yahoo Messenger}, \textbf{Google Talk} oraz serwisami opartymi na XMPP;
    \item[--] otwarte \textbf{API}, umożliwiające integrację z innymi tego typu aplikacjami.
\end{itemize}

W 2000 roku na rynek wprowadzono aplikację typu open source o nazwie \textbf{Jabber} oraz standardowy protokół \textbf{Extensible Messaging and Presence Protocol} (w skrócie XMPP). Serwery XMPP potrafiły działać jak bramy (ang. \textit{gateway}) dla innych protokołów wymiany wiadomości, w związku, z czym wystarczyła jedna aplikacja typu klient, aby móc korzystać z wielu sieci. Serwery XMPP obsługiwały wszystkie najpopularniejsze wówczas protokoły.


\section[Przegląd najpopularniejszych komunikatorów]{Przegląd najpopularniejszych komunikatorów}
\end{document}
