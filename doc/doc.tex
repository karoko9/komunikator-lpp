\documentclass[a4paper,12pt]{article}
\usepackage{fancyhdr}
\usepackage[MeX]{polski}
\usepackage[utf8]{inputenc}
\usepackage{amsmath,amsthm,amssymb}
\usepackage{graphicx}

\usepackage[lmargin=2.7cm]{geometry}

\pagestyle{fancy}
\lhead{Komunikator internetowy -- Dokumentacja}
\rhead{\bfseries}



\author{Adrian Chudziński, Przemysław Gospodarczyk}
\title{Komunikator internetowy -- Dokumentacja}

\begin{document}

\makeatletter
    \renewcommand\@seccntformat[1]{\csname the#1\endcsname.\quad}
    \renewcommand\numberline[1]{#1.\hskip0.7em}
\makeatother

\begin{titlepage}
\begin{center}

    \textsc{Licencjacki projekt programistyczny}\\[0.1cm]
    \textsc{IIUWr 2009/2010}\\[6cm]
    Adrian Chudziński, Przemysław Gospodarczyk\\[1cm]
    \textsc{\Large Komunikator internetowy}\\[0.25cm]
    \textsc{\large Dokumentacja}\\[8.675cm]

    {\footnotesize
    \begin{tabular}{| c | p{4cm} | p{4.25cm} | c | }
        \hline
        Wersja  &
        Zmiany  &
        Autorzy &
        Data    \\
        \hline
        1.0                                                                   &
        Pierwsza wersja                                                       &
        \par Adrian Chudziński \par Przemysław Gospodarczyk                                                  &
        2010-03-14                                                            \\
        \hline
    \end{tabular}
    }

\end{center}
\end{titlepage}

\break

\setcounter{page}{2}

\tableofcontents

\break
\section[Słownik pojęć]{Słownik pojęć}
\section[Wstęp, komunikatory jako zjawisko społeczne]{Wstęp, komunikatory jako zjawisko społeczne}
Podstawowymi cechami Internetu są globalność i masowość.
Głównym elementem Internetu jest społeczność, która traktuje go nierzadko jako jedyne okno na świat.
Użytkownicy, korzystając z Internetu w poszukiwaniu informacji, danych lub szukając kontaktu z innymi ludźmi w świecie wirtualnym nie tylko rozwijają globalną sieć, ale stają sie również jej integralną i nieodzowną częścią.
Człowiek jako istota społeczna uzewnętrznia swoją potrzebę komunikacji poprzez pośrednią i bezpośrednią wymianę informacji z innymi ludźmi. Komunikacja sieciowa jest pośrednia i pozbawiona komunikatów niewerbalnych takich jak:
ton głosu, mimika, wygląd i zachowanie. Wyżej wymienione aspekty komunikacji sieciowej jednoznacznie wskazują na jej ubogość w stosunku do bezpośrednich kontaktów międzyludzkich. Niewerbalna komunikacja może wzmacniać, osłabiać, a nawet zaprzeczać przekazom werbalnym. W zwiazku z tym komunikaty werbalne mają ogromne znaczenie w komunikacji i mogą stać się ważniejsze niż treść wypowiedzi.

\par Komunikacja sieciowa posiada jednak swoje zalety, które przez wiele osób są marginalizowane.
Najważniejszą cechą komunikacji internetowej jest jej anonimowość. Użytkownicy często fałszują swoje dane osobowe lub nie podają ich w ogóle. Możliwa jest błyskawiczna zmiana tożsamości. Bezpośrednia komunikacja w świecie rzeczywistym jest znacznie trudniejsza niż internetowa. Ludzie hamowani przez nieśmiałość oraz lęk przed odrzuceniem w dużej mierze rezygnują z tego typu komunikacji. Anonimowość w sieci, często złudna powoduje, że wyżej wymienione czynniki tracą na znaczeniu. Badania japońskich naukowców z Ochanomizu University przeprowadzone na studentach dowodzą, że relacje interpersonalne w życiu wymagają pewnych umiejętności społecznych i orientacji na innych.

\par Komunikatory internetowe stały się bardzo popularną formą komunikacji wśród młodzieży, grupy społecznej która najbardziej potrzebuje kontaktów z rówieśnikami oraz akceptacji i dowartościowania. Nie każdy nastolatek jest wyposażony w wymagany przez komunikację bezpośrednią zestaw cech. Komunikatory internetowe ich nie wymagają. Każdy użytkownik może przedstawić siebie w zupełnie inny, wymarzony przez siebie sposób, koloryzując rzeczywistość. Nastolatkowie utrzymujacy takie kontakty zwiększają swoją pewność siebie, która będzie procentować w życiu dorosłym.

\par Komunikacja ludzi dorosłych [todo...]
  
\par Komunikatory internetowe pomagają w życiu społecznym osobom niepełnosprawnym, ludziom, którzy z powodu swojego kalectwa posiadają niskie zdanie o sobie i obawiają się odrzucenia z powodu swojej odmienności. W Internecie, chory człowiek nie obawia się zdemaskowania prawdy o sobie ani o swojej ewentualnej nieatrakcyjności fizycznej spowodowanej kalectwem.

\par Istnieje wiele teorii na temat wypierania kontaktów w świecie rzeczywistym przez kontakty w Internecie, a wpływ komunikatorów internetowych na życie człowieka w społeczeństwie jest oceniany przez naukowców bardzo niejednoznacznie.\\
Socjolog Sherry Turkle w swojej książce \emph{Life on the Screen} udowodniła, że rozwój komunikatorów internetowych i portali społecznościowych prowadzi najpierw do wyobcowania poszczególnych jednostek, a w konsekwencji do społecznego autyzmu i nadmiernego indywidualizmu internautów.
Według autorki internauci stają się leniwi, nie podejmują trudów związanych z utrzymywaniem kontaktów społecznych w świecie rzeczywistym.\\
Z drugiej strony, według kanadyjskiego socjologa Barry'ego Wellmana uznawanego powszechnie za jednego z najwybitniejszych badaczy współczesnej cywylizacji, w tym także Internetu, komunikacja sieciowa uzupełnia kontakty nie eliminując jednocześnie podstawowych form komunikacji, np. spotkań bezpośrednich. Według jego badań internaci częściej interesują sie polityką i biorą udział w życiu społecznym. 

\par Fakt, że w dobie elektronicznej pandemii gadulstwa autorzy projektu tworzą kolejne narzędzie do internetowego plotkowania należy usprawieliwić ogromnym zainteresowaniem tego typu aplikacjami oraz, co za tym idzie ogromnymi pieniędzmi za reklamy. [todo...]
\section[Historia komunikatorów]{Historia komunikatorów}
\section[Przegląd najpopularniejszych komunikatorów]{Przegląd najpopularniejszych komunikatorów}
\end{document}
