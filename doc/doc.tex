\documentclass[a4paper,12pt]{article}
\usepackage{fancyhdr}
\usepackage[MeX]{polski}
\usepackage[utf8]{inputenc}
\usepackage{amsmath,amsthm,amssymb}
\usepackage{graphicx}

\usepackage[lmargin=2.7cm]{geometry}

\pagestyle{fancy}
\lhead{Komunikator internetowy -- Dokumentacja}
\rhead{\bfseries}



\author{Adrian Chudziński, Przemysław Gospodarczyk}
\title{Komunikator internetowy -- Dokumentacja}

\begin{document}

\makeatletter
    \renewcommand\@seccntformat[1]{\csname the#1\endcsname.\quad}
    \renewcommand\numberline[1]{#1.\hskip0.7em}
\makeatother

\begin{titlepage}
\begin{center}

    \textsc{Licencjacki projekt programistyczny}\\[0.1cm]
    \textsc{IIUWr 2009/2010}\\[6cm]
    Adrian Chudziński, Przemysław Gospodarczyk\\[1cm]
    \textsc{\Large Komunikator internetowy}\\[0.25cm]
    \textsc{\large Dokumentacja}\\[8.675cm]

    {\footnotesize
    \begin{tabular}{| c | p{4cm} | p{4.25cm} | c | }
        \hline
        Wersja  &
        Zmiany  &
        Autorzy &
        Data    \\
        \hline
        1.0                                                                   &
        Pierwsza wersja                                                       &
        \par Adrian Chudziński \par Przemysław Gospodarczyk                                                  &
        2010-03-14                                                            \\
        \hline
    \end{tabular}
    }

\end{center}
\end{titlepage}

\break

\setcounter{page}{2}

\tableofcontents

\break
\section[Słownik pojęć]{Słownik pojęć}
\textbf{Komunikator internetowy} (ang. \textit{Instant messenger}) -- program komputerowy pozwalający na przesyłanie natychmiastowych komunikatów (ang. \textit{Instant messaging}) pomiędzy dwoma lub więcej komputerami, poprzez sieć Internet. Od poczty elektronicznej różni się tym, że oprócz samej wiadomości, przesyłane są także informacje o obecności użytkowników, co zwiększa znacznie szansę na prowadzenie bezpośredniej konwersacji.\\

\noindent\textbf{Serwis społecznościowy (portal społecznościowy)} -- rodzaj interaktywnych stron WWW, które są współtworzone przez sieci społeczne osób podzielających wspólne zainteresowania lub chcących poznać zainteresowania innych. Większość portali społecznościowych dostarcza użytkownikom wielu sposobów komunikacji, np.: czaty, komunikatory, listy dyskusyjne, blogi i fora dyskusyjne.\\

\noindent\textbf{Twitter} -- darmowy serwis społecznościowy udostępniający usługę mikroblogowania umożliwiającą użytkownikom wysyłanie oraz odczytywanie tak zwanych tweetów. Tweet to krótka, nie przekraczająca 140 znaków wiadomość tekstowa wyświetlana na stronie użytkownika oraz dostarczana pozostałym użytkownikom, którzy obserwują dany profil. Użytkownicy mogą dodawać krótkie wiadomości do swojego profilu z poziomu strony głównej serwisu, wysyłając SMS-y lub korzystając z zewnętrznych aplikacji.\\

\section[Wstęp, komunikatory jako zjawisko społeczne]{Wstęp, komunikatory jako zjawisko społeczne}
Podstawowymi cechami Internetu są globalność i masowość.
Głównym elementem Internetu jest społeczność, która traktuje go nierzadko jako jedyne okno na świat.
Użytkownicy, korzystając z Internetu w poszukiwaniu informacji, danych lub szukając kontaktu z innymi ludźmi w świecie wirtualnym nie tylko rozwijają globalną sieć, ale stają się również jej integralną i nieodzowną częścią.
Człowiek jako istota społeczna uzewnętrznia swoją potrzebę komunikacji poprzez pośrednią i bezpośrednią wymianę informacji z innymi ludźmi. Komunikacja sieciowa jest pośrednia i pozbawiona komunikatów niewerbalnych takich jak:
ton głosu, mimika, wygląd i zachowanie. Wyżej wymienione aspekty komunikacji sieciowej jednoznacznie wskazują na jej prymitywność w stosunku do bezpośrednich kontaktów międzyludzkich. Niewerbalna komunikacja może wzmacniać, osłabiać, a nawet zaprzeczać przekazom werbalnym. W związku z tym komunikaty werbalne mają ogromne znaczenie w komunikacji i mogą stać się ważniejsze niż treść wypowiedzi. Internet powoduje, że wzajemne kontakty stają się bardziej powierzchowne i tracą ważną część swojego charakteru.

\par Komunikacja sieciowa posiada jednak swoje zalety, które przez wiele osób są marginalizowane.
Najważniejszą cechą komunikacji internetowej jest jej anonimowość. Użytkownicy często fałszują swoje dane osobowe lub nie podają ich w ogóle. Możliwa jest błyskawiczna zmiana tożsamości. Bezpośrednia komunikacja w świecie rzeczywistym jest znacznie trudniejsza niż internetowa. Ludzie hamowani przez nieśmiałość oraz lęk przed odrzuceniem w dużej mierze rezygnują z tego typu komunikacji. Anonimowość, często złudna powoduje, że wyżej wymienione czynniki tracą na znaczeniu. Badania japońskich naukowców z Ochanomizu University przeprowadzone na studentach dowodzą, że relacje interpersonalne w życiu wymagają pewnych umiejętności społecznych oraz orientacji na innych.

\par Komunikatory internetowe stały się bardzo popularną formą komunikacji wśród młodzieży, grupy społecznej, która najbardziej potrzebuje kontaktów z rówieśnikami oraz akceptacji i dowartościowania. Współczesna młodzież jest bardziej spragniona relacji interpersonalnych niż pokolenie urodzone w latach sześćdziesiątych i siedemdziesiątych.
Według Stevena Gerali, szefa chicagowskiego \emph{Department of Youth Ministry \& Adolescent Studies} współcześni młodzi ludzie chcą intymności, stałości oraz bezpiecznych związków z innymi. Przede wszystkim pragną być doceniani i chronieni. Ważnym czynnikiem staje się anonimowość, która zapewnia młodemu użytkownikowi bezpieczeństwo oraz do pewnego stopnia bezkarność i łatwość w poznawaniu oraz komunikacji z ludźmi z całego świata.
W Internecie zawiera się przyjaźnie, przeżywa miłość, zdobywa informacje i nowe hobby.\\
Nie każdy nastolatek jest wyposażony w wymagany przez komunikację bezpośrednią zestaw cech. Komunikatory internetowe ich nie wymagają. Każdy użytkownik może przedstawić siebie w zupełnie inny, wymarzony przez siebie sposób, koloryzując rzeczywistość. Nastolatkowie utrzymujący takie kontakty zwiększają swoją pewność siebie, która będzie procentować w życiu dorosłym.

\par Wbrew powszechnym opiniom, z komunikatorów internetowych i portali społecznościowych korzystają coraz częściej ludzie dorośli, których przeciąga darmowość usługi oraz prostota obsługi. W artykule \emph{Rosja też ćwierka}, z numeru jedenastego czasopisma \emph{Polityka}, autorzy Marek Ostrowski i Adam Szostkiewicz opisują portal społecznościowy Twitter jako miejsce wymiany zdań głównie między ludźmi pełnoletnimi.\\
Za niespodziankę można uznać wielką popularność Twittera, zważywszy na to, że umożliwia on wysyłanie wiadomości o długości do 140 znaków, co powinno stanowić poważne utrudnienie w korespondowaniu. Użytkownicy Twittera nie wykorzystują go jednak do poważnych rozmów, lecz do bezzwłocznego pisania o tym co ich w danym momencie interesuje i emocjonuje. Ponadto serwis pozwala na sprawdzenie, co interesuje innych oraz według psychologa z Uniwersytetu Warszawskiego Jana Zająca sprzyja rozmowom o niczym w celu podtrzymania relacji towarzyskich z innymi ludźmi.\\
Opinie o tym, że komunikacja internetowa cofa nas w czasie do ery sprzed rewolucji Gutenberga i powoduje kryzys prasy drukowanej są mocno przesadzone. Według pobieżnej analizy przeprowadzonej przez autorów wspomnianego artykułu, prasa jest ciągle bardziej opiniotwórcza i dostarcza więcej informacji.\\
Twitter sprzyja również rozwojowi literatury mądrości (ang. \textit{wisdom literature}), czyli pisaniu aforyzmów, czego nie można już uznać za prymitywną formę komunikacji. Wysyłane krótkich wiadomości nie tylko, więc zaśmieca głowę, ale również potrafi zmobilizować do myślenia.

\par Komunikatory internetowe pomagają w życiu społecznym osobom niepełnosprawnym, ludziom, którzy z powodu swojego kalectwa posiadają niskie zdanie o sobie oraz obawiają się odrzucenia z powodu swojej odmienności. W Internecie, chory człowiek nie obawia się zdemaskowania prawdy o sobie ani o swojej ewentualnej nieatrakcyjności fizycznej spowodowanej kalectwem.

\par Istnieje wiele teorii na temat wypierania kontaktów w świecie rzeczywistym przez kontakty w Internecie, a wpływ komunikatorów internetowych na życie społeczne człowieka jest oceniany przez naukowców bardzo niejednoznacznie.\\
Socjolog Sherry Turkle w swojej książce \emph{Life on the Screen} udowodniła, że rozwój komunikatorów internetowych i portali społecznościowych prowadzi najpierw do wyobcowania poszczególnych jednostek, a w konsekwencji do społecznego autyzmu i nadmiernego indywidualizmu internautów.
Według autorki internauci stają się leniwi, nie podejmują trudów związanych z utrzymywaniem kontaktów społecznych w świecie rzeczywistym.\\
Z drugiej strony, według kanadyjskiego socjologa Barry'ego Wellmana uznawanego powszechnie za jednego z najwybitniejszych badaczy współczesnej cywilizacji, w tym także Internetu, komunikacja sieciowa uzupełnia kontakty nie eliminując jednocześnie podstawowych form komunikacji, np. spotkań bezpośrednich. Według jego badań internaci częściej interesują się polityką i biorą udział w życiu społecznym.

\par Fakt, że w dobie elektronicznej pandemii gadulstwa autorzy projektu tworzą kolejne narzędzie do internetowego plotkowania należy usprawiedliwić ogromnym zainteresowaniem tego typu aplikacjami oraz, co za tym idzie ogromnymi pieniędzmi za reklamy. Przed napisaniem własnego oprogramowania, autorzy przeprowadzili analizę komunikatorów internetowych dostępnych na polskim rynku, aby nie powielać błędów konkurencji i wykorzystać najlepsze pomysły.

\section[Historia komunikatorów]{Historia komunikatorów}
\section[Przegląd najpopularniejszych komunikatorów]{Przegląd najpopularniejszych komunikatorów}
\end{document}
